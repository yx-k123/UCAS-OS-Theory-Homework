\documentclass[11pt]{article}

\usepackage[a4paper]{geometry}
\geometry{left=2.0cm,right=2.0cm,top=2.5cm,bottom=2.5cm}

\usepackage{ctex}
\usepackage{amsmath,amsfonts,graphicx,amssymb,bm,amsthm}
\usepackage{algorithm,algorithmicx}
\usepackage[noend]{algpseudocode}
\usepackage{fancyhdr}
\usepackage{listings}
\usepackage{tabularx}
\usepackage{booktabs}


\newcounter{counter_exm}\setcounter{counter_exm}{1}
%\newcounter{counter_thm}\setcounter{counter_thm}{1}
%\newcounter{counter_lma}\setcounter{counter_lma}{1}
%\newcounter{counter_dft}\setcounter{counter_dft}{1}
%\newcounter{counter_clm}\setcounter{counter_clm}{1}
%\newcounter{counter_cly}\setcounter{counter_cly}{1}

\newtheorem{theorem}{{\hskip 1.7em \bf 定理}}
\newtheorem{lemma}[theorem]{\hskip 1.7em 引理}
\newtheorem{proposition}[theorem]{Proposition}
\newtheorem{claim}[theorem]{\hskip 1.7em 命题}
\newtheorem{corollary}[theorem]{\hskip 1.7em 推论}
\newtheorem{definition}[theorem]{\hskip 1.7em 定义}

\renewcommand{\emph}[1]{\begin{kaishu}#1\end{kaishu}}

\newenvironment{solution}{{\noindent\hskip 2em \bf 解 \quad}}


\renewenvironment{proof}{{\noindent\hskip 2em \bf 证明 \quad}}{\hfill$\qed$\par}
\newenvironment{example}{{\noindent\hskip 2em \bf 例 \arabic{counter_exm}\quad}}{\addtocounter{counter_exm}{1}\par}

\newenvironment{concept}[1]{{\bf #1\quad} \begin{kaishu}} {\end{kaishu}\par}

\newcommand\E{\mathbb{E}}

\begin{document}
    
\pagestyle{fancy}
\lhead{\kaishu 中国科学院大学}
\chead{}
\rhead{\kaishu 2025年秋季学期操作系统理论课}
    
\begin{center}
    {\LARGE \bf 第一次作业}\\
\end{center}
    \begin{kaishu}
        \hfill 寇逸欣 2023K8009922004
    \end{kaishu}

\section{实验环境}
本实验在 WSL2 环境下进行,操作系统版本如下:

\begin{verbatim}
Linux LAPTOP-C42RC799 6.6.87.2-microsoft-standard-WSL2 #1 SMP PREEMPT_DYNAMIC
Thu Jun  5 18:30:46 UTC 2025 x86_64 x86_64 x86_64 GNU/Linux
\end{verbatim}

实验所用的硬件为一台具有 Intel Core i9 14900HX 处理器和 32GB 内存的笔记本电脑。

\section{代码实现}
本次实验的代码实现包括三种不同方式的系统调用:glibc封装的系统调用、直接使用syscall指令的系统调用以及使用内联汇编实现的系统调用。
代码文件分别命名为 \texttt{open\_glibc.c}、\texttt{open\_syscall.c}、\texttt{open\_asm.c} 和 \texttt{getpid\_glibc.c}、\texttt{getpid\_syscall.c}、\texttt{getpid\_asm.c}。每
个文件中都包含了相应的系统调用实现以及测试代码。

其中计时函数选择了 \texttt{clock\_gettime},而不是 \texttt{gettimeofday},因为前者提供了更高的精度(纳秒级)和更稳定的时间测量,适合用于性能测试。而后者仅提供微秒级精度,在高精度需求场景下可能不够准确。
除此之外,\texttt{gettimeofday} 还可能受到系统时间调整的影响,导致测量结果不准确。

\section{实验结果}
实验结果如下表所示,表中列出了每种系统调用实现的平均执行时间(单位:纳秒)。每个测试用例均运行了 1000 次以确保结果的稳定性。实验对每一项
的测试设计了100次预热,以减少缓存和其他系统状态对测量结果的影响。

\begin{table}[H]
    \centering
    \renewcommand{\arraystretch}{1.5}
    \begin{tabularx}{0.8\textwidth}{l X X X}
        \toprule
        \textbf{函数/平均时间(ns)} & \textbf{test1} & \textbf{test2} & \textbf{test3} \\
        \midrule
        \texttt{open\_glibc}    & 1965 & 1974 & 2020 \\
        \texttt{open\_syscall}  & 2002 & 1974 & 2049 \\
        \texttt{open\_asm}      & 1185 & 1265 & 1261 \\
        \texttt{getpid\_glibc}  & 1293 & 1301 & 1279 \\
        \texttt{getpid\_syscall}& 251  & 248  & 255  \\
        \texttt{getpid\_asm}    & 198  & 200  & 193  \\
        \bottomrule
    \end{tabularx}
    \caption{不同系统调用实现的平均执行时间(单位:纳秒)}
\end{table}

\section{结果分析}
\subsection{针对同一个系统调用,记录和对比三种方法的运行时间,并分析时间差异原因}
对于 \texttt{open} 系统调用,glibc 封装版本和直接使用 syscall 指令版本的运行时间相差不大,均在 2000 纳秒左右。这是因为 glibc 封装的 \texttt{open} 函数本质上也是通过 syscall 指令实现的,因此性能差异较小。此外,\texttt{open} 是一个复杂的系统调用,涉及参数准备、路径解析、权限检查等多个步骤,glibc 的额外封装开销在整体流程中占比很小,所以两者时间差异不明显。然而,使用内联汇编实现的 \texttt{open} 系统调用显著更快,平均时间约为 1200 纳秒。这可能是因为内联汇编能够更直接地控制寄存器和指令,从而减少函数调用开销和上下文切换时间。

对于 \texttt{getpid} 系统调用,glibc 封装版本的运行时间约为 1300 纳秒,而直接使用 syscall 指令版本显著更快,平均时间约为 250 纳秒。使用内联汇编实现的 \texttt{getpid} 系统调用进一步提升了性能,平均时间约为 200 纳秒。这表明,对于简单的系统调用,直接使用 syscall 指令和内联汇编可以显著减少函数调用开销,从而提高性能。原因在于 \texttt{getpid} 作为一个极其简单的系统调用,glibc 封装虽然只是简单调用 syscall,但函数封装、参数传递和返回值处理等额外开销在总执行时间中占比较大,因此直接 syscall 和内联汇编版本明显更快。

\subsection{对比getpid和open这两个系统调用在使用内联汇编实现调用时的运行时间,如果有差异,尝试解释差异原因。若无差异,说明即可。}

在使用内联汇编实现系统调用时,\texttt{getpid} 和 \texttt{open} 这两个系统调用的运行时间存在显著差异。具体来说,\texttt{getpid} 的平均执行时间约为 200 纳秒,而 \texttt{open} 的平均执行时间约为 1200 纳秒。这种差异主要源于两个系统调用的复杂性和所需的资源。
\texttt{getpid} 是一个非常简单的系统调用,它只需要返回当前进程的 ID,涉及的操作非常少,因此执行时间较短。而 \texttt{open} 则是一个相对复杂的系统调用,它需要处理文件路径解析、权限检查以及文件描述符的分配等多个步骤,这些操作都需要更多的时间和资源。因此,\texttt{open} 系统调用的执行时间显著高于 \texttt{getpid}。

\end{document}





